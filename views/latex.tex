
\section{Introduction}
\TeX\ looks more difficult than it is. It is
almost as easy as $\pi$. See how easy it is to make special
symbols such as $\alpha$,
$\beta$, $\gamma$,
$\delta$, $\sin x$, $\hbar$, $\lambda$, $\ldots$ We also can make
subscripts
$A_{x}$, $A_{xy}$ and superscripts, $e^x$, $e^{x^2}$, and
$e^{a^b}$. We will use \LaTeX, which is based on \TeX\ and has
many higher-level commands (macros) for formatting, making
tables, etc. More information can be found in Ref.~\cite{latex}.

We just made a new paragraph. Extra lines and spaces make no
difference. Note that all formulas are enclosed by
\$ and occur in \textit{math mode}.

The default font is Computer Modern. It includes \textit{italics},
\textbf{boldface},
\textsl{slanted}, and \texttt{monospaced} fonts.

\section{Equations}
Let us see how easy it is to write equations.
\begin{equation}
\Delta =\sum_{i=1}^N w_i (x_i - \bar{x})^2 .
\end{equation}
It is a good idea to number equations, but we can have a
equation without a number by writing
\begin{equation}
P(x) = \frac{x - a}{b - a} , \nonumber
\end{equation}
